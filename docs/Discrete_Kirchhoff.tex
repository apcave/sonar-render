\documentclass{article}
\usepackage{amsmath}
\begin{document}

\section{Discrete Kirchhoff Scattering Equation}


\subsection{Spreading in the Near-Field}
When the distance between the source and the receiver is small or the source is not a mono pole but a fragment of area, the usual spreading term is not valid.
Instead ratios of areas can be used and given that a fragment of area is not infinitely small a couple of approximations are provided. Also in the near field for scattering from surface to surface half spherical spreading may be more appropriate.

The usual spreading term is related to intensity $I$ and the initial and final areas $S_s$ and $S_i$. The pressure units are Pa relative to a mono pole at 1 meter (Pa re @ 1 m).
\[
S_s = 4\cdot \pi \cdot r_s^2  = 4 \cdot \pi
\]
\[
S_i = 4 \cdot \pi \cdot r_{si}^2
\]
Assuming sperical spreading in free space the intesity at at distance is given by
\[
I_i = I_s (S_s/S_i)
\]
give that pressure is related to intensity by
\[
p_s = \sqrt{I_s \cdot \rho \cdot cp}
\]
\[
p_i = \sqrt{I_s \cdot (S_s/S_I) \cdot \rho \cdot cp}
\]
the pressure ratios determined by the ratio of the areas
\[
\frac{p_i}{p_s} = \sqrt{\frac{S_s}{S_i}} = \sqrt{\frac{4\cdot \pi}{4\cdot \pi \cdot r_{si}^2}} = \frac{1}{r_{si}}
\]
which reduces to the inverse of the distance. This is the usual spreading term used.

Now suppose that the starting intensity $I_s$ is provided by a small fragment of area $A_s$ replacing $S_s$ in the above equations. Pressure ratio becomes
\[
\frac{p_i}{p_s} = \sqrt{\frac{A_s}{S_i}} = \sqrt{\frac{A_s}{4\cdot \pi \cdot r_{si}^2}} = \frac{1}{r_{si}} \cdot \sqrt{\frac{A_s}{4\cdot \pi}}
\]
for spreading from a fragment of area $A_s$ into the far field. When processing reflections between two surfaces that are close half spherical spreading may be more appropriate. The pressure ratio becomes
\[
\frac{p_i}{p_s} = \frac{1}{r_{si}} \cdot \sqrt{\frac{A_s}{2\cdot \pi}}
\]
Now approximating as $\lim{r_{si} \to 0}$ and $S_i < A_s$ the intensity cannot magnify then replace  $S_i$ with $A_s$ and the pressure ratio becomes
\[
\frac{p_i}{p_s} = \sqrt{\frac{A_s}{A_s}} = 1
\]
Approximation when spreading for a fragment $A_s$ to a fragment of area $A_i$ and $\lim{r_{si} \to 0}$ if $S_i < A_i$ and $A_i \geq A_s$ the intensity needs to dissipate over destination surface fragment replace  $S_i$ with $A_s$ and the pressure ratio becomes
\[
\frac{p_i}{p_s} = \sqrt{\frac{A_s}{A_i}}
\]
or
\[
\frac{p_i}{p_s} = 1
\]
for this case if $A_i < A_s$.

\subsection{Mono-Pole Source to Surface}

When the source is a mono-pole 
\[
    p_{\text{inc}}(\mathbf{r}_i) = \frac{A_i e^{i \cdot k \cdot r_{si}}}{r_{si}}
\]

where:

\begin{itemize}
    \item \( r_{si} = |\mathbf{r}_i - \mathbf{r}_s| \)
    \item \( \mathbf{r}_i \) is the center of facet \( i \)
    \item \( \mathbf{r}_s \) is the source point
    \item \( A_i \) is the area of facet \( i \)
    \item \( k = \frac{2\pi}{\lambda} \) is the wavenumber
\end{itemize}

\subsection{Surface to Field Point}
The discrete Kirchhoff scattering equation for the scattered acoustic pressure at a field point \( \mathbf{r} \) is:

\[
p_{\text{scat}}(\mathbf{r}) = \sum_{i} 
\left[
    \frac{ik}{2\pi} \cdot 
    p_{\text{inc}}(\mathbf{r}_i) \cdot 
    \frac{e^{ik r_{ri}}}{r_{ri}} \cdot 
    (\hat{\mathbf{r}}_{ri} \cdot \mathbf{n}_i) \cdot 
    A_i
\right]
\]
applying the previous equations for spreading
\[
    p_{\text{scat}}(\mathbf{r}) = \sum_{i} 
    \left[
        \frac{ik}{2\pi} \cdot 
        p_{\text{inc}}(\mathbf{r}_i) \cdot 
        \frac{e^{ik r_{ri}}}{r_{ri}} \cdot
        \sqrt{\frac{A_i}{2 \pi}} \cdot
        (\hat{\mathbf{r}}_{ri} \cdot \mathbf{n}_i) \cdot 
        A_i
    \right]
\]



where:

\begin{itemize}
    \item \( \mathbf{r}_i \) is the center of facet \( i \)
    \item \( \mathbf{r}_{ri} = \mathbf{r} - \mathbf{r}_i \)
    \item \( r_{ri} = |\mathbf{r}_{ri}| \)
    \item \( \hat{\mathbf{r}}_{ri} = \frac{\mathbf{r}_{ri}}{r_{ri}} \)
    \item \( \mathbf{n}_i \) is the unit normal of facet \( i \)
    \item \( A_i \) is the area of facet \( i \)
    \item \( p_{\text{inc}}(\mathbf{r}_i) \) is the incident pressure at \( \mathbf{r}_i \)
    \item \( k = \frac{2\pi}{\lambda} \) is the wavenumber
\end{itemize}




The total scattered pressure at a receiver facet centered at \( \mathbf{r}_i \) due to contributions from all source facets \( j \) is:

\[
p_{\text{scat}}(\mathbf{r}_i) = \sum_{j}
\left[
\frac{ik}{2\pi} \cdot
p_{\text{inc}}(\mathbf{r}_j) \cdot
\frac{e^{ik r_{ij}}}{r_{ij}} \cdot
(\hat{\mathbf{r}}_{ij} \cdot \mathbf{n}_j) \cdot
(\hat{\mathbf{r}}_{ij} \cdot \mathbf{n}_i) \cdot
A_j
\right]
\]

where:

\begin{itemize}
    \item \( \mathbf{r}_i, \mathbf{r}_j \) are the centers of receiver and source facets
    \item \( \mathbf{r}_{ij} = \mathbf{r}_i - \mathbf{r}_j \), and \( r_{ij} = |\mathbf{r}_{ij}| \)
    \item \( \hat{\mathbf{r}}_{ij} = \frac{\mathbf{r}_{ij}}{r_{ij}} \)
    \item \( \mathbf{n}_i, \mathbf{n}_j \) are the unit normals of the receiver and source facets
    \item \( A_j \) is the area of the \( j \)-th source facet
    \item \( p_{\text{inc}}(\mathbf{r}_j) \) is the incident pressure at the \( j \)-th facet
    \item \( k = \frac{2\pi}{\lambda} \) is the wavenumber
\end{itemize}

\end{document}
